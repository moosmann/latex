% ********************************************************************
%%%%%%%%%%%%%%%%%%%%%%%%%%%%%%%%%%%%%%%%%%%%%%%%%%%%%%%%%%%%%%%%%%%%%%
% classicthesis-config.tex 
%*********************************************************************

% ********************************************************************
% ********************************************************************
% 1. Configure classicthesis for your needs here, e.g., 
% remove "drafting" below 
% in order to deactivate the time-stamp on the pages
% ****************************************************************************
\PassOptionsToPackage{eulerchapternumbers,listings,
  %drafting,%
  pdfspacing,%floatperchapter,%linedheaders,%
  subfig,beramono,eulermath,parts,subfig}{classicthesis}
% ********************************************************************
% Available options for classicthesis.sty:
% drafting  parts nochapters linedheaders
% eulerchapternumbers beramono eulermath pdfspacing minionprospacing
% tocaligned dottedtoc manychapters listings floatperchapter subfig
% ********************************************************************

% ********************************************************************
% Triggers for this config
% ******************************************************************** 
\usepackage{ifthen}
\newboolean{enable-backrefs} % enable backrefs in the bibliography
\setboolean{enable-backrefs}{false} % true false
% ****************************************************************************

% ****************************************************************************
% ****************************************************************************
% 2. Personal data and user ad-hoc commands
% ****************************************************************************
\newcommand{\myTitleGerman}{Nichtlinearer Zugang f{\"u}r das inverse Problem der Phasenr{\"u}ckgewinnung aus Einzelmessungen der R{\"o}ntgenintensit{\"a}ten bei nur einem Propagationsabstand\xspace}
\newcommand{\myTitle}{Nonlinear approaches to the inverse problem
  of phase retrieval from single-measurement X-ray intensity data\xspace}
\newcommand{\myDegree}{Dipl.-Phys.\xspace}
\newcommand{\myName}{Julian Philipp Moosmann\xspace}
\newcommand{\myProf}{Prof. Dr. Tilo Baumbach\xspace}
\newcommand{\mySupervisor}{Dr. habil. Ralf Hofmann (Privatdozent)\xspace}
\newcommand{\myFaculty}{Physik\xspace}
\newcommand{\myDepartment}{Laboratorium für Applikationen der
  Synchrotronstrahlung (LAS)/ Institut für Photonenforschung und
  Synchrotronstrahlung (IPS)\xspace} 
\newcommand{\myUni}{Karlsruher Institut für Technologie (KIT)\xspace}
\newcommand{\myLocation}{Karlsruhe\xspace}
\newcommand{\myTime}{09.01.2015\xspace}
\newcommand{\myVersion}{version 0.1\xspace}
%\newcommand{\myOtherProf}{Prof. Dr. Yankeedoodle\xspace}

% ********************************************************************
% Setup, finetuning, and useful commands
% ********************************************************************
\newcounter{dummy} % necessary for correct hyperlinks (to index, bib, etc.)
\newlength{\abcd} % for ab..z string length calculation
\providecommand{\mLyX}{L\kern-.1667em\lower.25em\hbox{Y}\kern-.125emX\@}
% ********************************************************************

% ********************************************************************
% ********************************************************************
% New and renewed command
% ********************************************************************
\newcommand{\ie}{i.\,e.}
\newcommand{\Ie}{I.\,e.}
\newcommand{\eg}{e.\,g.}
\newcommand{\Eg}{E.\,g.} 
% Use \boldsymbol instead of \mathbf to allow for bold greek letter.
% The package bm is more carefull, but changes font of bodl letter.
% Use \mathbold to use bodl fonts from eulermath
%\renewcommand{\vec}[1]{\mathbf{#1}}
%\renewcommand{\vec}{\boldsymbol}
\renewcommand{\vec}{\mathbold}
% Vectors
\newcommand{\xt}{{\vec{x},t}}
% 1d
\newcommand{\kp}{k_\perp}
\newcommand{\xip}{\xi_{\perp}}
\newcommand{\xipm}{\abs{\xi_{\perp}}_m}
\newcommand{\xin}{\xi_{\mathrm{noise}}}
\newcommand{\xim}{\xi_{m}}
\newcommand{\xic}{\xi_{\mathrm{c}}}
\newcommand{\xib}{\xi_{\mathrm{b}}}
\newcommand{\xir}{{\xi_{r}}}
\newcommand{\xis}{{\xi_{s}}}
\newcommand{\xix}{{\xi_{x}}}
\newcommand{\xiy}{{\xi_{y}}}
\newcommand{\xiz}{{\xi_{z}}}
% Vectors: 2D
\newcommand{\np}{\nabla_\perp}
\newcommand{\inp}{\nabla_\perp^{-2}}
\newcommand{\npa}{\nabla_{\perp,\alpha}}
\newcommand{\vecdp}{\vec{d}_\perp}
\newcommand{\veckp}{\vec{k}_\perp}
\newcommand{\vecqp}{\vec{q}_\perp}
\newcommand{\vecrp}{\vec{r}_\perp}
\newcommand{\vecxp}{\vec{x}_\perp}
\newcommand{\vecsp}{\vec{s}_\perp}
\newcommand{\vecxip}{\vec{\xi}_\perp}
% Vectors: 3D
\newcommand{\vecd}{\vec{d}}
\newcommand{\vecj}{\vec{j}}
\newcommand{\veck}{\vec{k}}
\newcommand{\vecn}{\vec{n}}
\newcommand{\vecp}{\vec{p}}
\newcommand{\vecq}{\vec{q}}
\newcommand{\vecr}{\vec{r}}
\newcommand{\vecs}{\vec{s}}
\newcommand{\vecv}{\vec{v}}
\newcommand{\vecx}{\vec{x}}
\newcommand{\x}{\vec{x}}
\newcommand{\vecxi}{\vec{\xi}}
\newcommand{\vecA}{\vec{A}}
\newcommand{\vecB}{\vec{B}}
\newcommand{\vecE}{\vec{E}}
\newcommand{\vecD}{\vec{D}}
\newcommand{\vecH}{\vec{H}}
\newcommand{\vecM}{\vec{M}}
\newcommand{\vecP}{\vec{P}}
\newcommand{\vecS}{\vec{S}}
% Other
\newcommand{\BP}{\operatorname{BP}}
\newcommand{\FBP}{\operatorname{FBP}}
\newcommand{\adu}{\operatorname{ADU}}
\newcommand{\dqe}{\operatorname{DQE}}
\newcommand{\sgn}{\operatorname{sgn}}
\newcommand{\snr}[1]{\operatorname{SNR_{#1}}}
\newcommand{\NA}{\mathrm{NA}}
\newcommand{\Nrings}{N_{\circledcirc}}
\newcommand{\Npix}{N_\mathrm{x}}
\newcommand{\fwhm}{\operatorname{FWHM}}
\newcommand{\tie}{\operatorname{\textcolor{blue}{TIE}}}
\newcommand{\ctf}{\operatorname{\textcolor{red}{CTF}}}
\newcommand{\qp}{\operatorname{\textcolor{green}{QP}}}
\newcommand{\qpc}{\operatorname{\textcolor{black}{QP_\mathrm{con}}}}
\newcommand{\conv}{\operatorname{\ast\ast}}
\newcommand{\Prop}{\mathcal{P}}
\newcommand{\order}{\mathcal{O}}
\newcommand{\N}{\mathcal{N}}
\newcommand{\sigb}{\sigma_\mathrm{b}}
\newcommand{\sigs}{\sigma_\mathrm{s}}
\newcommand{\sig}[1]{\sigma_\mathrm{#1}}
%\newcommand{\Seik}{\mathcal{S}}
\newcommand{\e}[1]{\mathrm{e}^{#1}}
\newcommand{\me}{m_e}
\newcommand{\ed}{\rho_{e}}
\newcommand{\Iin}{I^{\mathrm{(in)}}}
\newcommand{\Ibin}{\bar{I}^{\mathrm{(in)}}}
\newcommand{\Psin}{ \Psi^\mathrm{(in)} }
\newcommand{\subn}[1]{_\mathrm{#1}}
\newcommand{\submax}{_\mathrm{max}}
\newcommand{\supin}{^\mathrm{(in)}}
\newcommand{\supl}{^{(l)}}
\newcommand{\env}{^\mathrm{(env)}}
\newcommand{\ho}[1]{^{(#1)}}
\newcommand{\hon}[1]{^{(\mathrm{#1})}}
\newcommand{\vecjin}{\vec{j}^{\mathrm{(in)}}}
\newcommand{\vecjsc}{\vec{j}^{\mathrm{(s)}}}
\newcommand{\hook}{k_h}
\newcommand{\dpo}{\delta\phi_0}
\newcommand{\dpom}{\abs{\delta{\phi_0}}_{\mathrm{max}}}
\newcommand{\delo}{\delta_\omega}
\newcommand{\fov}{\mathrm{FOV}}
\newcommand{\lmaxg}{{l_{\mathrm{max},g}}}
\newcommand{\lmaxphi}{{l_{\mathrm{max},\phi}}}
\newcommand{\nlo}[1]{\text{\small{NLO}}_{#1}}
\newcommand{\lc}{l_\mathrm{c}}
%\newcommand{\textsub}[1]{\ensuremath{_{\mbox{\footnotesize #1}}}}
%\newcommand{\sqrd}{{\textsuperscript 2}}
\newcommand{\rar}{\rightarrow}
\newcommand{\lar}{\leftarrow}
\newcommand{\ra}{\right\rangle}
\newcommand{\la}{\left\langle}
\newcommand{\ii}{\mathrm{i}}
% modulus
\newcommand{\abs}[1]{\left| #1 \right|} % for absolute value
\newcommand{\absn}[1]{| #1 |} % for absolute value when | too big
% average 
\newcommand{\avg}[1]{\left\langle #1 \right\rangle} % for average
\newcommand{\avgn}[1]{\langle #1 \rangle} 
% mean abs
\newcommand{\avgabs}[1]{\left\langle\left|#1\right|\right\rangle} 
\newcommand{\avgabsn}[1]{\langle|#1|\rangle} 
\newcommand{\mabs}[1]{\overline{#1}} 
% Fourier operator and symbol
\newcommand{\F}{\operatorname{\mathcal{F}}}
\newcommand{\iF}{\operatorname{\mathcal{F}^{-1}}}
\newcommand{\FO}[1]{\operatorname{\mathcal{F}#1}}
\newcommand{\ft}[1]{\widehat{#1}}
%\newcommand{\ft}[1]{\widetilde{#1}}
% angular averaged fft
\newcommand{\aft}[1]{\widetilde{#1}} 
% quasiparticle modification
\newcommand{\qpft}[1]{\widehat{#1}^{\mathrm{(QP)}}}
% derivatives and integral measures
\newcommand{\pd}{\partial}
\newcommand{\difd}{\mathrm{d}}
\newcommand{\Int}{\int\limits}
\newcommand{\intd}{\!\mathrm{d}}
\newcommand{\Intd}[1]{\!\mathrm{d}#1\,}
\newcommand{\Intdd}[1]{\!\mathrm{d^2}#1\,}
\newcommand{\Intddd}[1]{\!\mathrm{d^3}#1\,}
\newcommand{\const}{\mathrm{constant}}
% Matrices
\newcommand{\vv}[2]{\left(\begin{array}{c}#1\\#2\end{array}\right)}
\newcommand{\vvv}[3]{\left(\begin{array}{c}#1\\#2\\#3\end{array}\right)}
\newcommand{\mmmm}[4]{\left(\begin{array}{cc}#1&#2\\#3&#4\end{array}\right)}
% Other
\renewcommand{\hbar}{\hslash}
\newcommand{\bra}[1]{\left\langle#1\right|}
\newcommand{\ket}[1]{\left|#1\right\rangle}
% parentheses, brackets, and braces
\newcommand{\lp}{\left(}
\newcommand{\rp}{\right)}
\newcommand{\lrp}[1]{\left(#1\right)}
\newcommand{\lrb}[1]{\left[#1\right]}
\newcommand{\lrbr}[1]{\left\{#1\right\}}
\newcommand{\lb}{\left[}
\newcommand{\rb}{\right]}
\newcommand{\expsb}[1]{\exp\left[#1\right]}
\newcommand{\expb}[1]{\exp\left[#1\right]}
\newcommand{\expp}[1]{\exp\left(#1\right)}

\renewcommand{\floatpagefraction}{.7}
%*******************************************************
%********************************************************************
% Hyphenation
%*******************************************************
% use \hyphenchar\font=\string"7F for words containing hyphen
%\showhyphens{single-distance}
\hyphenation{
hard-X-ray 
sin-gle-dis-tance 
Rönt-gen 
mo-no-chro-ma-tic
par-ax-ial
Helm-holtz
Fou-ri-er
two-body
par-a-bolic
par-a-me-trised
Ang-ström-quel-le
mi-cro-to-mo-graphy
tech-nique
tech-niques
pe-ne-tra-tion
flu-o-res-cence
}
%*******************************************************

% ****************************************************************************
% 3. Loading some handy packages
% ****************************************************************************
% ******************************************************************** 
% Packages with options that might require adjustments
% ******************************************************************** 
\PassOptionsToPackage{utf8}{inputenc}	% latin9 (ISO-8859-9) = latin1+"Euro sign"
 \usepackage{inputenc}				
\PassOptionsToPackage{british,ngerman}{babel} %american
 \usepackage{babel}		
\PassOptionsToPackage{backend=biber,style=alphabetic,maxbibnames=9,maxnames=4}{biblatex}
\usepackage{biblatex}			
\bibliography{Bibliography}
\usepackage[]{csquotes}
% CREATES PROBLEM WHEN USING BIBLATEX
%\PassOptionsToPackage{square,numbers}{natbib}
% \usepackage{natbib}				
\PassOptionsToPackage{fleqn}{amsmath}% math environments and more by the AMS 
 \usepackage{amsmath}
% ******************************************************************** 
% General useful packages
% ******************************************************************** 
\PassOptionsToPackage{T1}{fontenc} % T2A for cyrillics
 \usepackage{fontenc} % alle 256 Zeichen des europäischen Zeichenvorrates in
 % T1-Kodierung.: Eingebunden wird die T1-Version der
 % Computer-Modern-Schriftsippe mit \usepackage[T1]{fontenc}. Dadurch
 % ist auch sichergestellt, dass in einem PDF Umlaute gefunden werden.
 % Sie umfasst 256 Zeichen und unterstützt damit sowohl die west- als
 % auch die osteuropäischen Sprachen mit lateinischem Alphabet.
\usepackage{scrhack} % fix warnings when using KOMA with listings package     
\usepackage{xspace} % to get the spacing after macros right  
\usepackage{mparhack} % get marginpar right
\usepackage{fixltx2e} % fixes some LaTeX stuff 
%\renewcommand*{\acsfont}[1]{\textssc{#1}} % for MinionPro
% ****************************************************************************

% ****************************************************************************
% 4. Setup floats: tables, (sub)figures, and captions
% ****************************************************************************
\usepackage{tabularx} % better tables
	\setlength{\extrarowheight}{3pt} % increase table row height
\newcommand{\tableheadline}[1]{\multicolumn{1}{c}{\spacedlowsmallcaps{#1}}}
\newcommand{\myfloatalign}{\centering} % to be used with each float for alignment
\usepackage{caption}
\captionsetup{format=hang,font=small}
\usepackage{subfig}
\usepackage{url}
% ****************************************************************************

% ****************************************************************************
% 5. Setup code listings
% ****************************************************************************
\usepackage{listings} 
%\lstset{emph={trueIndex,root},emphstyle=\color{BlueViolet}}%\underbar} % for special keywords
\lstset{language=[LaTeX]Tex,%C++,
    keywordstyle=\color{RoyalBlue},%\bfseries,
    basicstyle=\small\ttfamily,
    %identifierstyle=\color{NavyBlue},
    commentstyle=\color{Green}\ttfamily,
    stringstyle=\rmfamily,
    numbers=none,%left,%
    numberstyle=\scriptsize,%\tiny
    stepnumber=5,
    numbersep=8pt,
    showstringspaces=false,
    breaklines=true,
    frameround=ftff,
    frame=single,
    belowcaptionskip=.75\baselineskip
    %frame=L
} 
% ****************************************************************************
% ****************************************************************************
% 5./6. Additional package and options. Put here because hyperref
% package should be loaded as late as possible
% ****************************************************************************
%\usepackage{lmodern}
\usepackage{amssymb}
%\usepackage{bm} % bold math fonts !! changes appearance of bold fonts
%\usepackage{latexsym} % symbols already provided by the amsfonts & amssymb 
%\usepackage{mathabx} % more symbols: widehat, widecheck, etc
%\usepackage{yhmath}
\usepackage{mathtools}
\usepackage{mathrsfs}% calligrafic fonts for physicists
\usepackage{courier} % used for displaying code in text: \texttt{CODE}
%\usepackage{gensymb}
\usepackage{siunitx} %!! definitions(\celsius,...) in package gensymb collide with siunitx
\usepackage{textcomp} % removes warning: LaTeX Font Warning: Font
                      % shape `OMS/pplj/m/n' undefined 
%\usepackage[on]{auto-pst-pdf} % use psfrag with pdflatex
%\usepackage{psfrag} % replace labels in ps file
% global command option for psfragfig
\PassOptionsToPackage{process=auto}{pstool}%[process=none]%auto,all
 \usepackage{pstool}
\usepackage{classicthesis} 
%\usepackage{epstopdf} % to use pdflatex with eps images
%\graphicspath{ {./Images/} } % set image search path
%\usepackage[clearempty]{titlesec}
%\usepackage{todonotes}
% multible bibliographies
%\usepackage{multibib}
%\newcites{pub}{Publications}
%\usepackage{bibentry}
%\nobibliography*%{Bibliography}
\usepackage{tikz}
% Speed up compiling by externalising  tikz figures, not working
% \usepackage{pgfplots}
% \usetikzlibrary{external}
% \tikzexternalize[prefix=tikz/]
%\tikzset{external/force remake}
\usetikzlibrary{decorations.pathmorphing} % for decorations, wavy lines (snake)
%\usetikzlibrary{decorations.markings}% to draw tangent lines
%\usepackage{rotating} % for sidwaysfigure
%\usepackage{wrapfig} % wrap text around figure
\usepackage{caption}
\captionsetup{format=plain} 
\usepackage{chemfig}
\usepackage{enumitem}
%\usepackage{longtable}
%\usepackage{rotating}
\usepackage{tabu}
% ****************************************************************************
% 6. PDFLaTeX, hyperreferences and citation backreferences
% ****************************************************************************
% ********************************************************************
% Using PDFLaTeX
% ********************************************************************
\PassOptionsToPackage{hyperfootnotes=false,pdfpagelabels}{hyperref}
\usepackage{hyperref}  % backref linktocpage pagebackref
% most package must be loaded before hyperref to work properly except,
% e.g. amsref, cleveref
\pdfcompresslevel=9
\pdfadjustspacing=1 
% should avoid warnings: found pdf version 1.5 but at most version 1.4 allowed
\pdfminorversion=5
\PassOptionsToPackage{pdftex}{graphicx}
\usepackage{graphicx}

% acronym must be loaded after hyperref that hyperlinks to acronyms
% work properly. However it still doesn't. Otherwise use option
% 'nohyperlinks': wenn hyperref geladen ist, wird die Verlinkung
% unterbunden without compiling produces warning and the acronym links
% jump to the first page
\usepackage[printonlyused,nohyperlinks,
%smaller
]{acronym} % handling acronyms
 \renewcommand{\bflabel}[1]{{#1}\hfill} % fix the list of acronyms

% clever referencing. MUST BE LOADED AFTER HYPERREF
\usepackage[english,capitalise]{cleveref}
% Options: capitalise: for capital of all initial label, otherwise use
% \Cref and \Crefrange; cleveref: for full label names

% ********************************************************************
% Setup the style of the backrefs from the bibliography
% (translate the options to any language you use)
% ********************************************************************
\newcommand{\backrefnotcitedstring}{\relax}%(Not cited.)
\newcommand{\backrefcitedsinglestring}[1]{(Cited on page~#1.)}
\newcommand{\backrefcitedmultistring}[1]{(Cited on pages~#1.)}
\ifthenelse{\boolean{enable-backrefs}}%
{%
		\PassOptionsToPackage{hyperpageref}{backref}
		\usepackage{backref} % to be loaded after hyperref package 
		   \renewcommand{\backreftwosep}{ and~} % separate 2 pages
		   \renewcommand{\backreflastsep}{, and~} % separate last of longer list
		   \renewcommand*{\backref}[1]{}  % disable standard
		   \renewcommand*{\backrefalt}[4]{% detailed backref
		      \ifcase #1 %
		         \backrefnotcitedstring%
		      \or%
		         \backrefcitedsinglestring{#2}%
		      \else%
		         \backrefcitedmultistring{#2}%
		      \fi}%
}{\relax}    

% ********************************************************************
% Hyperreferences
% ********************************************************************
\hypersetup{%
    %draft,	% = no hyperlinking at all (useful in b/w printouts)
    colorlinks=true, 
    linktocpage=true, 
    pdfstartpage=3, 
    pdfstartview=FitV,%
    % uncomment the following line if you want to have black links (e.g., for printing)
    %colorlinks=false, linktocpage=false, pdfborder={0 0 0}, pdfstartpage=3, pdfstartview=FitV,% 
    breaklinks=true, 
    pdfpagemode=UseNone, 
    pageanchor=true, 
    pdfpagemode=UseOutlines,%
    plainpages=false, 
    bookmarksnumbered, 
    bookmarksopen=true, 
    bookmarksopenlevel=1,%
    hypertexnames=true, 
    pdfhighlight=/O, 
    % nesting=true,
    % frenchlinks,%
    urlcolor=webbrown, 
    linkcolor=RoyalBlue, 
    citecolor=webgreen, 
    % pagecolor=RoyalBlue,%
    %urlcolor=Black, linkcolor=Black, citecolor=Black, 
    %pagecolor=Black,%
    pdftitle={\myTitle},%
    pdfauthor={\textcopyright\ \myName, \myUni, \myFaculty},%
    pdfsubject={},%
    pdfkeywords={},%
    pdfcreator={pdfLaTeX},%
    pdfproducer={LaTeX with hyperref and classicthesis}%
}   

% ********************************************************************
% Setup autoreferences
% ********************************************************************
% There are some issues regarding autorefnames
% http://www.ureader.de/msg/136221647.aspx
% http://www.tex.ac.uk/cgi-bin/texfaq2html?label=latexwords you have
% to redefine the makros for the language you use, e.g., american,
% ngerman (as chosen when loading babel/AtBeginDocument)
% ********************************************************************

\makeatletter \@ifpackageloaded{babel}%
{%
  \addto\extrasamerican{%
    \renewcommand*{\figureautorefname}{Figure}%
    \renewcommand*{\tableautorefname}{Table}%
    \renewcommand*{\partautorefname}{Part}%
    \renewcommand*{\chapterautorefname}{Chapter}%
    \renewcommand*{\sectionautorefname}{Section}%
    \renewcommand*{\subsectionautorefname}{Section}%
    \renewcommand*{\subsubsectionautorefname}{Section}%
  }%
  \addto\extrasngerman{%
    \renewcommand*{\paragraphautorefname}{Absatz}%
    \renewcommand*{\subparagraphautorefname}{Unterabsatz}%
    \renewcommand*{\footnoteautorefname}{Fu\"snote}%
    \renewcommand*{\FancyVerbLineautorefname}{Zeile}%
    \renewcommand*{\theoremautorefname}{Theorem}%
    \renewcommand*{\appendixautorefname}{Anhang}%
    \renewcommand*{\equationautorefname}{Gleichung}%
    \renewcommand*{\itemautorefname}{Punkt}%
  }%

  % Fix to getting autorefs for subfigures right
  \providecommand{\subfigureautorefname}{\figureautorefname}%
    }{\relax}
\makeatother

