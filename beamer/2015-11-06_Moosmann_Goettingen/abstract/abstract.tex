\documentclass{article}

\begin{document}

\section{X-ray phase-contrast in vivo tomography}


Four-dimensional imaging techniques are essential tools in biology to
understand the behaviour of cells during embryonic development.  Here,
we apply X-ray phase-contrast microtomography to capture the early
development of the optically opaque African clawed frog (Xenopus
laevis) over the course of time and in 3D. Xenopus embryos lack
conventional X-ray absorption contrast and act as pure-phase objects
for hard X-rays.  The wave front exiting the sample is thus
characterised by a 2D phase map representing the projection of the
object along the X-ray beam.  Employing quasi-monochromatic and
sufficiently spatially coherent X-rays, we make use of
propagation-based phase-contrast.

In Fresnel theory, we study the formation of 2D intensity contrast
upon free-space propagation from a given phase map and how phase
retrieval based on linear contrast transfer breaks down at strong
phase variations.  Using a single-distance intensity measurement only,
we devise a phase-retrieval method for moderately strong phase
variations which, due to large propagation distances, exhibits high
spatial resolution and contrast at low photon statistics.

We discuss constraints imposed by in vivo imaging and present results
from experiments on living Xenopus embryos.


% Important properties of linear
% contrast transfer, which are conserved for a wide range of phase
% variations and propagation distances, are exploited in order to 

%, an important vertebrate model organism, 

%  By visualisation,
% segmentation, volume balancing, and optical flow analysis, we were
% able to observe a transient structure, address a question regarding
% the formation of the archenteron cavity, track single cells, and
% analyse the dynamics of tissues and individual cells.

\newpage
\section{Ralf at KTH}


We study how a 2D phase map, representing the absorption-free
projection of an object along a hard, quasi-monochromatic and
sufficiently spatially coherent X-ray beam, is converted into a 2D map
of intensity variations upon free-space propagation over a distance z.

In Fresnel theory, the according Fourier-space expansion in the
strength of phase variations starts with a linear and local term which
is modified by an infinite series of non-local corrections. We show
simulationally and by an actual experiment that two important
properties of linear contrast transfer are conserved for a wide range
of nonlinear phase variations and propagation distances and how this
can be understood in terms of the breaking pattern of a phase-scaling
symmetry which is exact in the limit of vanishing phase variations. An
according phase-retrieval algorithm is devised which, thanks to large
values of z, exhibits high spatial resolution and contrast at low
photon statistics. Finally, we compare our results of 3D
reconstructed, time-lapsed electron density in living vertebrate model
embryos with those obtained by a conventional phase-retrieval
algorithm.

\end{document}
